\documentclass[headsepline,titlepage,twoside,12pt,toc=flat,headings=normal]{scrreprt}
\usepackage{nfdisoftware}
\usepackage{booktabs}% nice tables with \toprule, \midrule and \bottomrule
\usepackage{csquotes}% \enquote command
\usepackage{lipsum}% for example text, comment out when replaced
\usepackage{microtype}% optional, looks better but can be commented out if it causes errors e.g. on MiKTeX
\usepackage{natbib}
\usepackage{cleveref}%\cref command for floats like tables
\usepackage{rotating}
\usepackage{pifont} \newcommand{\cmark}{\ding{51}}

% Replace values ***
\newcommand{\thetitle}{Descriptive review of already existing solutions, plans and policies regarding software marketplaces in NFDI consortias}
\newcommand{\thedate}{\today}
\newcommand{\theauthor}{\orcid{Konrad Höffner}{0000-0001-7358-3217}}
\newcommand{\editor}{\orcid{Matthias Löbe}{0000-0002-2344-0426}}
\newcommand{\orcid}[2]{\href{https://orcid.org/#2}{#1}}
\newcommand{\contributor}{\orcid{Lisa Schwier}{0009-0006-0299-4813}, \orcid{Neelam Vishen}{0009-0004-8940-3312}, \orcid{Martin Reinhardt}{0000-0002-1213-5135} and \orcid{Beate Hetenyi}{0009-0006-0497-2327}. Structure, GitHub README and phrasing inspired by the \href{https://github.com/gesiscss/Jupyter4NFDI_survey_results}{Jupyter4NFDI User Survey Results} by Julian Kohne.}
\newcommand{\doi}{10.xxxx/xxxx}
\newcommand{\version}{1.0}
\newcommand{\licensename}{CC0 1.0 Universal}
\newcommand{\licenseuri}{https://creativecommons.org/publicdomain/zero/1.0/}
% *****************
\newcommand{\question}[1]{\subsubsection{#1}}
% multiple choice checkboxes
\usepackage{enumitem,amssymb}
\newlist{answers}{itemize}{2}
\setlist[answers]{label=$\square$}
\newcommand{\otherbox}{\fbox{\phantom{This is how big an answer would be.}}}
% *****************
\author{\theauthor}
\date{\thedate}
\title{\thetitle}
\subtitle{Deliverable 2.2}

\begin{document}

\maketitle
\imprint

%************************************************
\chapter*{Abstract}\label{ch:abstract}
\addcontentsline{toc}{chapter}{Abstract}

This report describes the existing state of software metadata and marketplaces within the NFDI as a prerequisite for the work in the nfdi.software project, which builds upon it.
To obtain an accurate and current picture of the current state, a survey was conducted withhin the NFDI.
In total, 140 people started the survey and 51 finished it between 03 March 2025 and the extended end date of 14 April 2025.
Of the 26 consortia, 23 had at least one representative answer the survey, with no answer from MaRDI, NFDI4Energy and NFDIxCS.
Within their consortia, participants fulfill various roles, but mostly management, research and software development distantly followed by administator roles and a very low amount of data stewards.
To get more users, software is disseminated by means of scientific publications closely followed by web pages and with more distance by social media.
To describe research software, \ac{CFF} is a clear favourite but Zenodo.json and CodeMeta are used as well.

Regarding research software market places, answers were nearly tied in a three-way split given in close order:
The first group has plans to build or use one but nothing concrete yet.
The second group does not have any current plans regarding software marketplaces but would like to stay informed.
The third group does not have any current plans regarding software marketplaces and are not interested in this topic.
A further signifincant portion of the respondends have linked or are currently linking their software metadata to an existing software marketplace.
A minority does not have any current plans regarding software marketplaces and is not interested in this topic.

To ensure the sustainability of their services and software, consortia rely on active communities, hope the community contributes or keep contributing after the projects end in this order.

All these rankings are based both on individual responses as well as those normalized by consortium with no significant changes in relative preference.

%************************************************

%************************************************
\chapter*{Abbreviations}% use like \ac{NFDI} in text
\addcontentsline{toc}{chapter}{Abbreviations}
%************************************************

\begin{acronym}
\acro{NFDI}{Nationale Forschungsdateninfrastruktur}
%\acro{RSE}{Research Software Engineering}
\acro{RSE}{Research Software Encyclopedia}
\acro{BERD}{(Big) Data in Business, Economics, and Related Research}
\acro{CFF}{Citation File Format}
\acro{GHGA}{German Human Genome-Phenome Archive}
\acro{MaRDI}{Mathematical Research Data Initiative}
\acro{PUNCH}{Particles, Universe, NuClei and Hadrons}
\acro{FAIR}{Findable, Accessible, Interoperable, Reusable}
\acro{FAIRmat}[FAIRmat]{FAIR data infrastructure for condensed-matter physics and the chemical physics of solids}
\acro{FAIRagro}[FAIRagro]{FAIR research data management system for the agrosystems research community}
\end{acronym}
\acused{NFDI}

\setcounter{tocdepth}{1}% no subsection in TOC
\tableofcontents
\addcontentsline{toc}{chapter}{\contentsname}
%\pdfbookmark[1]{\contentsname}{tableofcontents}

%************************************************
%\chapter{Introduction}\label{ch:introduction}
%************************************************

%************************************************
\chapter{Survey Description}\label{ch:questionnaire}
%************************************************
The original text elements to motivate and introduce the survey are given below.
%\section{Survey on software marketplace usage}

\section{Description}
\subsection{English}
This survey is conducted as part of the nfdi.software project to gain insights into existing research software marketplaces, metadata, and sustainability strategies within the NFDI consortia.
The responses will contribute to Deliverable D2.2 as well as other project-related activities.
The survey questions are in English but this description and the data protection guidelines at the end are also translated to German.

\subsection{German}
Diese Umfrage wird im Rahmen des nfdi.software-Projekts durchgeführt, um Einblicke in bestehende Forschungssoftware-Marktplätze, Metadaten und Nachhaltigkeitsstrategien innerhalb der NFDI-Konsortien zu gewinnen.
Die Antworten fließen in das Deliverable D2.2 sowie in weitere projektbezogene Aktivitäten ein.

\section{Welcome Message}
nfdi.software is a Base4NFDI funded project that plans to design, implement and test a central marketplace for research software in Germany which provides aid in finding software collected from distributed metadata repositories (e.g. RSE, physics.tools), related publications and code repositories (BRE).
The aim is to improve the accessibility of NFDI research data and its access with specific software packages and enable more complex (re)use of metadata and software.
The need for a central access portal for research software of the NFDI consortia arises from the growing need of the scientific disciplines of the cultural sciences, humanities and social sciences, engineering and the natural and life sciences to ensure the sustainable use and further development of research software.
nfdi.software is intended to link and coordinate independent individual developments from these areas in a federated data infrastructure.
The networking and contextualisation of research software opens up the potential to use research data sustainably and to expand the spectrum of analysis and processing of research data.

We are currently in the initialization phase, so now is your chance to tell us about your existing research software, metadata, your specific needs and potentially existing marketplaces and registries!
Your responses are critical to design our infrastructure and prevent duplicate work, and we will share the evaluated responses with you, so please fill out the short survey (5-10 minutes)!

\section{Data Protection Declaration}\label{english-data-protection-declaration}

\subsection{English}

\subsubsection{What Data We Collect \& Why}
\emph{General survey responses} (aggregated and anonymized) are used for project evaluation and may be included in reports or publications.

\emph{Optional personal data} (name, email address) is only collected if you voluntarily provide it.
This allows us to follow up in case of any uncertainties.

\subsubsection{How Your Data is Used \& Shared}
Survey results are analyzed only in aggregated and anonymized form.
Individual responses will not be publicly linked to any person.
Consortium names may be published in reports or publications, but Individual names will remain confidential unless you explicitly agree to their inclusion.
Names and email addresses are used solely for internal inquiries and will not be shared with third parties.

\subsubsection{Data Storage \& Retention}\label{data-storage-retention}
Survey responses will be securely stored and deleted once it is no longer required for the project.
If results are used in publications, this will be done in a fully anonymized manner, except for consortium names where applicable.

\subsubsection{Your Rights \& Contact Information}\label{your-rights-contact-information}
Participation is voluntary, and you may withdraw your consent at any time.
If you wish to request the deletion of your personal data, please contact us at \href{mailto:konrad.hoeffner@uni-leipzig.de}{konrad.hoeffner@uni-leipzig.de}.
By completing the survey, you confirm that you have read and understood this privacy statement.

\subsection{German}

\subsubsection{Welche Daten wir erheben \& warum}\label{welche-daten-wir-erheben-warum}
Allgemeine Umfrageantworten (aggregiert und anonymisiert) werden zur Projektbewertung genutzt und können in Berichten oder Publikationen einfließen.

Optionale personenbezogene Daten (Name, E-Mail-Adresse) werden nur erhoben, wenn Sie diese freiwillig angeben.
Dies ermöglicht uns, bei Unklarheiten gezielt nachzufragen.
\subsubsection{Wie Ihre Daten verwendet und weitergegeben werden}
Die Umfrageergebnisse werden nur in aggregierter und anonymisierter Form analysiert.
Einzelne Antworten werden nicht öffentlich einer Person zugeordnet.
Konsortiennamen können in Berichten und Publikationen veröffentlicht werden aber aber Namen von Personen können nur dann veröffentlicht werden, wenn Sie dem explizit zustimmen.
Namen und E-Mail-Adressen werden ausschließlich intern für Rückfragen verwendet und nicht an Dritte weitergegeben.


\subsubsection{Speicherung \& Aufbewahrung der Daten}
Die Umfrageantworten werden sicher gespeichert und wenn sie nicht mehr gebraucht werden gelöscht.
Falls die Ergebnisse in Publikationen genutzt werden, erfolgt dies in vollständig anonymisierter Form mit Ausnahme gegebenenfalls der Namen von Konsortien.


\subsubsection{Ihre Rechte \& Kontaktmöglichkeiten}
Die Teilnahme ist freiwillig, und Sie können Ihre Einwilligung jederzeit widerrufen.
Wenn Sie die Löschung Ihrer personenbezogenen Daten beantragen möchten, kontaktieren Sie uns unter konrad.hoeffner@uni-leipzig.de.

\emph{Mit dem Ausfüllen der Umfrage bestätigen Sie, dass Sie diese Datenschutzerklärung gelesen und verstanden haben.}

\chapter{Questions and Answers}
\begin{table}[ht!]
\begin{tabulary}{\textwidth}{lLL}
\toprule
\emph{Code}              &\emph{Question}   &\emph{Type}\\
\midrule
consortium        &Which consortium do you report the solutions for?    &Multiple choice\\
solutions         &Which solutions and plans does your consortium have regarding research software marketplaces?    &Multiple choice\\
sustain           &How do you intend to ensure the sustainability (maintaining usability and further development) of the services and software developed or utilized in your consortium?    &Multiple choice with comments\\
disseminate       &How do you disseminate your software to get more users?    &Multiple choice\\
metadata          &Which metadata formats do you describe your research software with (if any)?    &Multiple choice\\
ontologies        &If you use controlled vocabularies, ontologies or knowledge graphs to describe your research software, please list them below (separated via commas, if multiple).    &Short free text\\
sourcecode        &Where do you publish your software and its source code?    &Multiple choice\\
forward           &In case you cannot answer some of the questions for your consortium and you know someone who can, please give us their email address below.  &Short free text\\
contact           &Thank you for filling out the survey, it is really important for us and helps us a lot! Feel free to give us your full name and email in case we have some requests for clarification. &Long free text\\
\bottomrule
\end{tabulary}
\caption{All questions of the questionnaire.}
\label{tab:questions}
\end{table}

The questions, including their answer options and answer distribution, are shown below.
\Cref{tab:questions} provides a list of these questions.

%\subsection{Consortium and Role}\label{consortium-and-role}
\section{Consortium}

\question{Which consortium do you report the solutions for?}

Please choose \emph{all} that apply:

\begin{answers}
\item \acs{BERD}
\item DAPHNE4NFDI
\item DataPlant
\item \acs{FAIRagro}
\item \acs{FAIRmat}
\item \acs{GHGA}
\item KonsortSWD
\item \acs{MaRDI}
\item NFDI4Biodiversity
\item NFDI4BioImage
\item NFDI4Cat
\item NFDI4Chem
\item NFDI4Culture
\item NFDI4DataScience
\item NFDI4Earth
\item NFDI4Energy
\item NFDI4Health
\item NFDI4Immuno
\item NFDI4Ing
\item NFDI-MatWerk
\item NFDI4Memory
\item NFDI4Microbiota
\item NFDI4Objects
\item NFDIxCS
\item \acs{PUNCH}4NFDI
\item Text+
\item Other: \otherbox
\end{answers}

\begin{figure}[h!]
\includegraphics[width=\textwidth]{plot/consortium.pdf}
\caption{}
\label{fig:consortium}
\end{figure}

Free form \emph{other} answers:
\begin{itemize}
\item Research Software Task Force OA7
\item Storage4PUNCH
\end{itemize}

\newpage
\section{Role}

\question{What are your roles in the consortium?}
Please choose \emph{all} that apply:

\begin{answers}
\item Software Developer
\item Researcher
\item Data Steward
\item Administrator or Software/Service Provider
\item Consortium/Task area/ Task Lead, Project manager, \ldots{}
\item Unsure/Don't know
\item Other: \otherbox
\end{answers}

\begin{figure}[h!]
\includegraphics[width=\textwidth]{plot/role.pdf}
\caption{What are your roles in the consortium?}
\label{fig:role}
\end{figure}

\begin{figure}[h!]
\includegraphics[width=\textwidth]{plot/role_n.pdf}
\caption{What are your roles in the consortium? (normalized)}
\label{fig:role_n}
\end{figure}

\begin{table}
\caption{Role pairs.}
\label{tab:role_pairs}
\input{table_role_pairs.tex}
\end{table}

\begin{table}
\caption{Role combinatinos.}
\label{tab:role_combinations}
\input{table_role_combinations.tex}
\end{table}

Free form \emph{other} answers:
\begin{itemize}
\item Service Steward
\item Technical Advisory Board member
\item Data Publisher
\end{itemize}

%\subsection{Main}\label{main}
\section{Solutions}
\question{Which solutions and plans does your consortium have regarding research software marketplaces?}

Comment only when you choose an answer.
Please choose \emph{all} that apply and provide a comment:

\begin{answers}
\item We have or are currently building our own software marketplace
\item We have or are currently linking our software metadata to an existing software marketplace.
\item We have some plans to structure our software / metadata but nothing concrete yet.
\item We do not have any current plans regarding software marketplaces but would like to stay informed.
\item We do not have any current plans regarding software marketplaces and are not interested in this topic.
\item Other: \otherbox
\end{answers}

Feel free to elaborate about your existing solutions in the text fields
next to the answers.

\begin{figure}[h!]
\includegraphics[width=\textwidth]{plot/solutions.pdf}
\caption{}
\label{fig:solutions}
\end{figure}

\begin{figure}[h!]
\includegraphics[width=\textwidth]{plot/solutions_n.pdf}
\caption{}
\label{fig:solutions_n}
\end{figure}

Free form comments:
\begin{itemize}
\item Is on our forecast but no specific integration plan.
\item Our software illustrates the design of an API. Maybe it can be used as a template for developers.
\item open for concrete suggestions
\item some developments from other NFDI consortia might be interesting for us to bring own tools and software on the other market place
\item We hope to use nfdi.software as market place for our own code and also to recommend its use as a good practice during tutorials
\item streamlining with bio.tools and Bioschemas profiles
\item We have different tools and services we would like to offer to a wider audience. Our involved workflow system Galaxy pretty much already handles the market place stuff.
\item we are working on a metadata schema
\item decentral indexing of software based on standardized metadata files
\item registering our software application 
\item Currently we have software created and coming from different institutes with their own repositories or marketplaces. We probably won't change dramatically this structure, however for software and metadata created for the consortium specifically we still need to decide how to structure and manage the repositories and marketplace.
\item We plan to link our software metadata to the nfdi.software RSD and fill the RSD with entries regarding Earth System sciences
\item You don't say what counts as market place. We use GitHub and GitHub-Zenodo integration.
\item started linking or updating tool information and including it in this domain, but ran in to problems due to domain editing rights being restricted to 1 user, solution not found: https://bio.tools/t?domain=nfdi4microbiota 
\item CRAN, PyPi, SSC (Stata Boston Repository), Zenodo
\item SSH Open Marketplace as registry for curating/disseminating services/tools/training materials: https://marketplace.sshopencloud.eu/
\item GHGA Software is distributed via Github.com, see https://github.com/ghga-de . It is also linked to the NFDI GitHub repository (https://github.com/nfdi-de).
\item We mainly use opensource projects that are hosted on e.g. Github
\item https://helmholtz.software/software/dcache
\item At the moment, this is just an overview website showcasing all the services relevant for our individual task areas.
\item we have a gitHub organisation (NFDI4Chem)
\item I'm not sure what you mean by marketplace (as we don't sell anything), but we have out website/dashboard in which we have our offered services
\item (but others are more qualified to report on it than I am)
\item It does not go under 'Marketplace' but its Open Source and we're prepared to collaborate
\end{itemize}

\section{Sustainability}
\question{How do you intend to ensure the sustainability (further development) of the services and software developed or utilized in your consortium?}

Please choose \emph{all} that apply:

\begin{answers}
\item None, when the research projects end, the software or service will stop working the next time it breaks.
\item We stop working on it when the projects end but it's
  open source.
We hope the community (e.g. developers on GitHub)
  contribute fixes and updates and we at least have someone who
  regularly checks and merges pull requests.
\item We actually have active communities maintaining and developing the
  software.
\item We ourselves keep contributing even after the projects end.
\item Other: \otherbox
\end{answers}

\begin{figure}[h!]
\includegraphics[width=\textwidth]{plot/sustain.pdf}
\caption{}
\label{fig:sustain}
\end{figure}

\begin{figure}[h!]
\includegraphics[width=\textwidth]{plot/sustain.pdf}
\caption{}
\label{fig:sustain_n}
\end{figure}

Free form \emph{other} answers:
\begin{itemize}
\item The approaches may differ depending on the specific service
\item unsure
\item can't tell
\item DataPLANT advocates an open contribution and collaboration model. We are enabling the community to work following open source software development principles. We are trying to setup some basic framework to support that.
\item At least in parts commitment from participating institutions to further develop and maintain services and software beyond the life-time of the consortium. In addition, there's the hope/wish to develop some semblance of community to ensure widespread use and development beyond the consortium.
\item we attach to existing sustainable projects instead of developing new solutions 
\item We provide a living document and a demonstrator, which is usable beyond project ending.
\item Depends on the service/software
\item We do not build software, we curate metadata (swMATH)
\item I don't know.
\item Don't know
\item There is work being done to ensure sustainability, but I'm personally not aware of any concrete details. 
\item But that's of course not true for all software; but software that already existed before punch4nfdi started is likely to be continued
\end{itemize}


\section{Dissemination}
\question{How do you disseminate your software to get more users?}\label{how-do-you-disseminate-your-software-to-get-more-users}
Please choose \emph{all} that apply:

\begin{answers}
\item Social Media
\item Institute, university or project web page
\item Scientific publication
\item Other: \otherbox
\end{answers}

\begin{figure}[h!]
\includegraphics[width=\textwidth]{plot/disseminate.pdf}
\caption{}
\label{fig:disseminate}
\end{figure}

\begin{figure}[h!]
\includegraphics[width=\textwidth]{plot/disseminate_n.pdf}
\caption{}
\label{fig:disseminate_n}
\end{figure}

Free form \emph{other} answers:
\begin{itemize}
\item Conferences, Talks
\item Public services like GitHUB
\item Conferences, workshops, summer/winter schools,...
\item conference presentations and expert discussions
\item Events, conferences, summer schools...
\item trainings
\item Conferences
\item project website, marketplace
\item zbmath.org
\item Tutorials, conferences
\item Scientific community communication
\item personal contacts, active acquisition
\item Annual User Workshops
\end{itemize}

\section{Metadata}
\question{Which metadata formats do you describe your research software with (if any)?}\label{which-metadata-formats-do-you-describe-your-research-software-with-if-any}
Please choose \emph{all} that apply:

\begin{answers}
\item CodeMeta, see \url{https://codemeta.github.io/}
\item \acf{CFF},\\see \url{https://citation-file-format.github.io/}
\item Zenodo.json, see \url{https://developers.zenodo.org/}
\item Other: \otherbox
\end{answers}

\begin{figure}[h!]
\includegraphics[width=\textwidth]{plot/metadata.pdf}
\caption{}
\label{fig:metadata}
\end{figure}

\begin{figure}[h!]
\includegraphics[width=\textwidth]{plot/metadata_n.pdf}
\caption{}
\label{fig:metadata_n}
\end{figure}

Free form \emph{other} answers:
\begin{itemize}
\item looking at \url{https://bioschemas.org/profiles/ComputationalTool/1.0-RELEASE}
\item unsure, as I am not involved in the actual development
\item other discipline specific sidecar files, dep5, oer metadata
\item Software-specific documetnation (e.g. R Package documentation using roxygen)
\item All kind, you can provide a configuration, and you use our design for data exchange and describing your data.
\item We don't set a policy yet, although software development guidelines are in the workplan.
\item We do not have that across all software we have, however this might be useful to add on our repository.
\item We are using a metadata schema based on international standards for genomic data sharing. The schema is modelled using LinkML and exists as a YAML. See https://github.com/ghga-de/ghga-metadata-schema
\item Don't know
\item NEXUS
\item none
\item GitHub
\end{itemize}


\section{Ontologies}
\question{If you use controlled vocabularies, ontologies or knowledge graphs to describe your research software, please list them below (separated via commas, if multiple).}
Please write your answer here: \otherbox

Free form \emph{other} answers:
\begin{itemize}
\item Whatever is used by established formats (ORCID, ROR, DCT).
\item EDAM,bioschemas,RO-Crate
\item EDAM, others may apply
\item unsure as I am not involved in the actual development
\item we have a long list of ontologies on our Ontology Worldmap (\url{https://github.com/nfdi4cat/Ontology-Overview-of-NFDI4Cat}) with more than 40 core ontologies
\item DataCite, Openaire, CodeMeta
\item TaDiRAH
\item Schema.org, maSMP vocabulary
\item Ontologies used are listed here: \url{https://docs.ghga.de/metadata/standards/#ontologies}. They include BRENDA Tissue Ontology, Data Use Ontology, Human Ancestry Ontology, Human Phenotype Ontology, International Classification of Diseases, Mondo Disease Ontology, National Cancer Institute Thesaurus.
\item We use our own ontology (\url{https://nfdi4earth.de/ontology/})
\item data management, FAIR, storage, data access, long-term archival, open-data
\end{itemize}

\section{Source Code}
\question{Where do you publish your software and its source code?}
Please choose \emph{all} that apply:

\begin{answers}
\item We do not create any research software.
\item We create research software but do not share the source code with others.
\item Public GitLab repositories.
\item Public GitHub repositories.
\item Not sure/don't know
\item Other: \otherbox
\end{answers}

\begin{figure}[h!]
\includegraphics[width=\textwidth]{plot/sourcecode.pdf}
\caption{}
\label{fig:sourcecode}
\end{figure}

\begin{figure}[h!]
\includegraphics[width=\textwidth]{plot/sourcecode_n.pdf}
\caption{}
\label{fig:sourcecode_n}
\end{figure}

Free form \emph{other} answers:
\begin{itemize}
\item Approaches may differ depending on service
\item Zenodo (given 4 times)
\item Under the nf-core project
\item GitLab; not sure how public/not public
\item Onw [sic] public gerrit-Server (git-based)
\item GH-Repo linked with Zenodo
\end{itemize}

\subsection{Final Questions}\label{final-questions}

\question{In case you cannot answer some of the questions for your consortium and you know someone who can, please give us their email address below.}

Please write your answer here: \otherbox

\paragraph{}
If you are unable to answer some of the questions for your consortium but know someone who can, you may voluntarily provide their email address below.
Please ensure that you have their consent before sharing their contact information.

\question{Thank you for filling out the survey, it is really important for us and helps us a lot! Feel free to give us your full name and email in case we have some requests for clarification.}
Please write your answer here: \otherbox

\paragraph{}
If you are providing someone else's contact information (e.g., a colleague who can help clarify certain points), please ensure you have their consent before sharing their name and email address with us.

\paragraph{}
\emph{Submit your survey.}

\paragraph{}
Thank you for completing this survey.

%************************************************
%\chapter{Results}\label{ch:results}
%************************************************

\newgeometry{margin=2cm}
\clearpage
\begin{sidewaystable}
\caption{Complete answers without comments.}
\label{tab:results}
\tiny
\setlength{\tabcolsep}{2pt}
\input{table_results.tex}
\end{sidewaystable}
\restoregeometry

%\section{Consortium}
%\section{Role}
%\section{Dissemination}
%\section{Metadata}
%\section{Solutions}
%\section{Source Code}

%************************************************
%\chapter{Recommendations}\label{ch:introduction}
%************************************************

%************************************************
%\chapter{References}\label{ch:introduction}
%************************************************

%\bibliographystyle{plainnat}%Options: "plainnat" for English text, "dinat" for German text to conform to DIN 1505 norm.
%%\bibliography{beispiel}

%************************************************
%\chapter{Appendix}\label{ch:appendix}
%************************************************

\end{document}
